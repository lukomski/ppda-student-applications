\documentclass[conference]{IEEEtran}
\IEEEoverridecommandlockouts
% The preceding line is only needed to identify funding in the first footnote. If that is unneeded, please comment it out.
\usepackage{cite}
\usepackage{amsmath,amssymb,amsfonts}
\usepackage{algorithmic}
\usepackage{graphicx}
\usepackage{textcomp}
\usepackage{xcolor}
\def\BibTeX{{\rm B\kern-.05em{\sc i\kern-.025em b}\kern-.08em
    T\kern-.1667em\lower.7ex\hbox{E}\kern-.125emX}}
\begin{document}

\title{Comparison of three selected ML models for predicting decision of the Dean}

\author{\IEEEauthorblockN{inż. Jan Łukomski}
\IEEEauthorblockA{\textit{Faculty of Electrical Engineering} \\
\textit{Warsaw Univerity of Technology}\\
Warsaw, Poland}
\and
\IEEEauthorblockN{inż. Paweł Podgórski}
\IEEEauthorblockA{\textit{Faculty of Electrical Engineering} \\
\textit{Warsaw Univerity of Technology}\\
Warsaw, Poland}
}

\maketitle

\begin{abstract}
%% TODO: abstract
\end{abstract}

\begin{IEEEkeywords}
ml, svm, knn, voting classifier
\end{IEEEkeywords}

\section{Introduction}
Predicting the decision of the Dean in an academic institution is a task of significant importance, as it can greatly influence the lives of students, faculty, and the overall direction of the institution. This article presents a comparative analysis of three carefully selected ML models - Support Vector Machines (SVM), Voting Classifier, and k-nearest neighbors (KNN) - to determine their effectiveness in predicting the Dean's decision as either positive or negative. By examining the performance, strengths, and limitations of each model, this study aims to provide valuable insights that can enhance decision-making processes within academic institutions.

\section{Data cleaning}
\subsection{Decision}
Our first step was to recognize if the decision of Dean was positive or negative.
From all of text representation of decision, we  choose  as positive decisions: 'Zamkniêty - decyzja pozytywna', 'Zamkniêty - inne', 'Decyzja pozytywna - oczekuje na realizacjê' and as negative desicions: 'Zamkniêty - decyzja negatywna', 'Zamkniêty - odrzucony formalnie', 'Zwrócony do korekty'.
We made a new column of type boolean where we indicate if decision was positive or negative.
\subsection{Data transformation}
There is 42 columns in dataset. Most of columns data types are \textit{string} like in "Mode of study", "Language", "Type of application", and some of them are of type \textit{float64}, e.g. "Attachments", "How many changes of statuses", "Missing ECTS".




\begin{thebibliography}{00}
\bibitem{b1} here put  biblio

\end{thebibliography}
\end{document}
